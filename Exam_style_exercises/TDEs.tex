% Options for packages loaded elsewhere
\PassOptionsToPackage{unicode}{hyperref}
\PassOptionsToPackage{hyphens}{url}
%
\documentclass[
]{article}
\usepackage{amsmath,amssymb}
\usepackage{iftex}
\ifPDFTeX
  \usepackage[T1]{fontenc}
  \usepackage[utf8]{inputenc}
  \usepackage{textcomp} % provide euro and other symbols
\else % if luatex or xetex
  \usepackage{unicode-math} % this also loads fontspec
  \defaultfontfeatures{Scale=MatchLowercase}
  \defaultfontfeatures[\rmfamily]{Ligatures=TeX,Scale=1}
\fi
\usepackage{lmodern}
\ifPDFTeX\else
  % xetex/luatex font selection
\fi
% Use upquote if available, for straight quotes in verbatim environments
\IfFileExists{upquote.sty}{\usepackage{upquote}}{}
\IfFileExists{microtype.sty}{% use microtype if available
  \usepackage[]{microtype}
  \UseMicrotypeSet[protrusion]{basicmath} % disable protrusion for tt fonts
}{}
\makeatletter
\@ifundefined{KOMAClassName}{% if non-KOMA class
  \IfFileExists{parskip.sty}{%
    \usepackage{parskip}
  }{% else
    \setlength{\parindent}{0pt}
    \setlength{\parskip}{6pt plus 2pt minus 1pt}}
}{% if KOMA class
  \KOMAoptions{parskip=half}}
\makeatother
\usepackage{xcolor}
\usepackage[margin=1in]{geometry}
\usepackage{color}
\usepackage{fancyvrb}
\newcommand{\VerbBar}{|}
\newcommand{\VERB}{\Verb[commandchars=\\\{\}]}
\DefineVerbatimEnvironment{Highlighting}{Verbatim}{commandchars=\\\{\}}
% Add ',fontsize=\small' for more characters per line
\usepackage{framed}
\definecolor{shadecolor}{RGB}{248,248,248}
\newenvironment{Shaded}{\begin{snugshade}}{\end{snugshade}}
\newcommand{\AlertTok}[1]{\textcolor[rgb]{0.94,0.16,0.16}{#1}}
\newcommand{\AnnotationTok}[1]{\textcolor[rgb]{0.56,0.35,0.01}{\textbf{\textit{#1}}}}
\newcommand{\AttributeTok}[1]{\textcolor[rgb]{0.13,0.29,0.53}{#1}}
\newcommand{\BaseNTok}[1]{\textcolor[rgb]{0.00,0.00,0.81}{#1}}
\newcommand{\BuiltInTok}[1]{#1}
\newcommand{\CharTok}[1]{\textcolor[rgb]{0.31,0.60,0.02}{#1}}
\newcommand{\CommentTok}[1]{\textcolor[rgb]{0.56,0.35,0.01}{\textit{#1}}}
\newcommand{\CommentVarTok}[1]{\textcolor[rgb]{0.56,0.35,0.01}{\textbf{\textit{#1}}}}
\newcommand{\ConstantTok}[1]{\textcolor[rgb]{0.56,0.35,0.01}{#1}}
\newcommand{\ControlFlowTok}[1]{\textcolor[rgb]{0.13,0.29,0.53}{\textbf{#1}}}
\newcommand{\DataTypeTok}[1]{\textcolor[rgb]{0.13,0.29,0.53}{#1}}
\newcommand{\DecValTok}[1]{\textcolor[rgb]{0.00,0.00,0.81}{#1}}
\newcommand{\DocumentationTok}[1]{\textcolor[rgb]{0.56,0.35,0.01}{\textbf{\textit{#1}}}}
\newcommand{\ErrorTok}[1]{\textcolor[rgb]{0.64,0.00,0.00}{\textbf{#1}}}
\newcommand{\ExtensionTok}[1]{#1}
\newcommand{\FloatTok}[1]{\textcolor[rgb]{0.00,0.00,0.81}{#1}}
\newcommand{\FunctionTok}[1]{\textcolor[rgb]{0.13,0.29,0.53}{\textbf{#1}}}
\newcommand{\ImportTok}[1]{#1}
\newcommand{\InformationTok}[1]{\textcolor[rgb]{0.56,0.35,0.01}{\textbf{\textit{#1}}}}
\newcommand{\KeywordTok}[1]{\textcolor[rgb]{0.13,0.29,0.53}{\textbf{#1}}}
\newcommand{\NormalTok}[1]{#1}
\newcommand{\OperatorTok}[1]{\textcolor[rgb]{0.81,0.36,0.00}{\textbf{#1}}}
\newcommand{\OtherTok}[1]{\textcolor[rgb]{0.56,0.35,0.01}{#1}}
\newcommand{\PreprocessorTok}[1]{\textcolor[rgb]{0.56,0.35,0.01}{\textit{#1}}}
\newcommand{\RegionMarkerTok}[1]{#1}
\newcommand{\SpecialCharTok}[1]{\textcolor[rgb]{0.81,0.36,0.00}{\textbf{#1}}}
\newcommand{\SpecialStringTok}[1]{\textcolor[rgb]{0.31,0.60,0.02}{#1}}
\newcommand{\StringTok}[1]{\textcolor[rgb]{0.31,0.60,0.02}{#1}}
\newcommand{\VariableTok}[1]{\textcolor[rgb]{0.00,0.00,0.00}{#1}}
\newcommand{\VerbatimStringTok}[1]{\textcolor[rgb]{0.31,0.60,0.02}{#1}}
\newcommand{\WarningTok}[1]{\textcolor[rgb]{0.56,0.35,0.01}{\textbf{\textit{#1}}}}
\usepackage{graphicx}
\makeatletter
\def\maxwidth{\ifdim\Gin@nat@width>\linewidth\linewidth\else\Gin@nat@width\fi}
\def\maxheight{\ifdim\Gin@nat@height>\textheight\textheight\else\Gin@nat@height\fi}
\makeatother
% Scale images if necessary, so that they will not overflow the page
% margins by default, and it is still possible to overwrite the defaults
% using explicit options in \includegraphics[width, height, ...]{}
\setkeys{Gin}{width=\maxwidth,height=\maxheight,keepaspectratio}
% Set default figure placement to htbp
\makeatletter
\def\fps@figure{htbp}
\makeatother
\setlength{\emergencystretch}{3em} % prevent overfull lines
\providecommand{\tightlist}{%
  \setlength{\itemsep}{0pt}\setlength{\parskip}{0pt}}
\setcounter{secnumdepth}{-\maxdimen} % remove section numbering
\ifLuaTeX
  \usepackage{selnolig}  % disable illegal ligatures
\fi
\IfFileExists{bookmark.sty}{\usepackage{bookmark}}{\usepackage{hyperref}}
\IfFileExists{xurl.sty}{\usepackage{xurl}}{} % add URL line breaks if available
\urlstyle{same}
\hypersetup{
  pdftitle={TDE},
  hidelinks,
  pdfcreator={LaTeX via pandoc}}

\title{TDE}
\author{}
\date{\vspace{-2.5em}2023-09-25}

\begin{document}
\maketitle

\hypertarget{exam-style-questions}{%
\section{Exam style questions}\label{exam-style-questions}}

\hypertarget{algorithmic-instructions}{%
\subsection{Algorithmic instructions}\label{algorithmic-instructions}}

\begin{itemize}
\tightlist
\item
  All the numerical values required need to be put on an A4 sheet and
  uploaded, alongside the required plots.
\item
  For all computations based on permutation/resampling, as well as split
  conformal, use B = 1000 replicates, and seed = 1991.
\item
  Both for confidence and prediction intervals, as well as tests, set α
  = 0.05.
\item
  When reporting a test result, please specify H0, H1 , the P−value and
  the corresponding conclusion.
\item
  When reporting confidence/prediction intervals, always provide upper
  and lower bound.
\end{itemize}

\hypertarget{exercise-1}{%
\subsubsection{Exercise 1}\label{exercise-1}}

Dr.~Yorkhamikis has been long retired from the academy. Time has
definitely passed since he left his Associate Professor job at a
prestigious university in Milan to go back to his hometown in Greece,
where he currently raises cattle. He still keeps his quantitative
approach though, and has instructed his employee to gather data relative
to three key quantities concerning the milk of his cows (filename
\(milk_samples_1.Rds\)) Assume observations to be independent between
them.

\begin{enumerate}
\def\labelenumi{\arabic{enumi}.}
\item
  Dr.~Yorkhamikis distrusts his employee, and is convinced there are
  observations that were not written down correctly and thus are not
  coherent with the true data. Devise a graphic method to retrieve such
  entries looking at the three measures relative to the milk and report
  the incoherent values.
\item
  The distrust of the Dr.~has been increasing a lot lately. He now also
  suspects that the second half of the observations were invented by his
  employee, who supposedly preferred to make the data up and have a nap.
  Dr.~Yorkhamikis says this will be evident in the median of the PH of
  the second half of the observations, which should be quite different
  from the median of the industry standard which is \(n\). Implement a
  pertinent (two-sided) statistical test relative to the median of the
  PH of the second half of the observations and write down your
  conclusions.
\item
  Repeat the test for the .25th quantile, using for \(H_0\) the .25th
  quantile equals the one of the first half (first 160-something rows)
  of the population.
\end{enumerate}

\begin{Shaded}
\begin{Highlighting}[]
\CommentTok{\#setwd("Exam\_style\_exercises/") \# Important to access files}
\CommentTok{\#load(file = "Exam\_style\_exercises/data/milk\_samples\_1.Rds") what happens if you run this line?}
\NormalTok{df.latte }\OtherTok{=} \FunctionTok{readRDS}\NormalTok{(}\AttributeTok{file =} \StringTok{"data/milk\_samples\_1.Rds"}\NormalTok{)}

\CommentTok{\# set "global" variables}
\end{Highlighting}
\end{Shaded}


\end{document}
